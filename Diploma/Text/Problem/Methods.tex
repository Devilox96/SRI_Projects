Для решения систем дифференциальных уравнений применяется множество численных методов, среди которых метод 
конечных разностей (FDM), метод конечных объёмов (FVM) и метод конечных элементов (FEM), используемые для 
решения гидродинамических задач. Рассмотрим каждый из них подробнее.

\begin{itemize}
	\item Метод конечных разностей заключается в разложении исследуемых функций по Тейлору для аппроксимации
	их производных, фактически сводя систему дифференциальных уравнений к системе (или серии систем)
	алгебраических уравнений. 
	
	\begin{equation*}
		f(x_0 + \dx) = f(x_0) + \frac{1}{1!}\dx f'(x_0) + \frac{1}{2!}\dx^2 f''(x_0) + ...
	\end{equation*}
	\begin{equation*}
		\displaystyle
		f'(x_0) = \frac{f(x_0 + \dx) - f(x_0)}{\dx} - \frac{1}{2!}\dx f''(x_0) - ... =
		\frac{f(x_0 + \dx) - f(x_0)}{\dx} + o(\dx)
	\end{equation*}
	
	Таким образом вычисление значений функции происходит в узлах расчётной сетки \cite{FDM}. Преимуществами такого 
	подхода являются простота реализации, возможность выбора различных независимых методов для интегрирования по 
	времени и пространству и увеличение порядка аппроксимации путём использования большего числа узлов сетки. 
	Недостаток метода в затруднительном описании сложных геометрий на всей исследуемой области и в области резких 
	перепадов производных функций.
	
	\item Метод конечных объёмов же рассматривается в некоторых ограниченных "секциях" расчётной области, 
	соприкасающиеся друг с другом. Возьмём некоторое гиперболическое уравнение и проинтегрируем его по 
	произвольному объёму:
	
	\begin{equation*}
		\frac{\partial u}{\partial t} + div(\overline{f}) = S(u)
	\end{equation*}
	\begin{equation*}
		\int\limits_V \frac{\partial u}{\partial t} dV + \int\limits_V div(\overline{f}) dV = \int\limits_V S(u) dV
	\end{equation*}
	
	Пользуясь теоремой Остроградского-Гаусса, получим закон сохранения:
	
	\begin{equation*}
		\int\limits_V div(\overline{f}) dV = \oint\limits_{\partial V} (\overline{f} \cdot \overline{n}) d \sigma
	\end{equation*}
	\begin{equation*}
		\int\limits_V \frac{\partial u}{\partial t} dV + \oint\limits_{\partial V} (\overline{f} \cdot \overline{n}) 
		d \sigma = 
		\int\limits_V S(u) dV
	\end{equation*}
	
	Теперь для численного решения необходимо дискретизировать данное уравнение. Искомые величины усредняются по 
	конечным объёмам:
	
	\begin{equation*}
		<u_i> = \frac{1}{V_i} \int\limits_{V_i}^{}u dV
	\end{equation*}
	\begin{equation*}
		\frac{1}{V_i} \int\limits_{V_i} S(u) dV = S(<u_i>) + O(h^2)
	\end{equation*}
	\begin{equation*}
		\oint\limits_{\sigma_k} (\overline{f} \cdot \overline{n}) d \sigma = 
		\overline{f}_k \cdot \overline{n}_k \sigma_k + O(h^p),
	\end{equation*}
	
	где $\sigma$ - площадь грани (длина отрезка) $k$, $\overline{n}_k$ - внешняя единичная нормаль к грани 
	(отрезку) $k$, $\overline{f}_k$ - значение потока в центре $k$, а $h$ - расстояние между центрами масс 
	конечных объёмов. Тогда схема приобретает вид
	
	\begin{equation*}
		\frac{\partial <u_i>}{\partial t} + \frac{1}{V_i} \sum_{k = 1}^{N_i} \overline{f}_k \cdot \overline{n}_k 
		\sigma_k = S(<u_i>).
	\end{equation*}
	
	Для окончательного построения какой-либо схемы необходимо определить подход к вычислению $\overline{f}_k$ и
	аппроксимации по времени \cite{FVM}.
\end{itemize}

\noindent Существуют ещё метод конечных элементов (FEM) и его комбинация с методом конечных объёмов (DGFEM), однако из-за
сложности описания и реализации оставим их за пределами данной работы \cite{FEM}.

Поскольку мы будем рассматривать кольцо атмосферы, заключённое между широтами $20^\circ$ и $70^\circ$, что
исключает возможность наличия сложной геометрии расчётной области, для простоты реализации воспользуемся методами
класса конечных разностей.
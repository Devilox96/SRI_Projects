Введём понятие устойчивости. Пусть есть некоторая дифференциальная краевая задача

\begin{equation*}
	Lu = f	
\end{equation*}

\noindent и составленная для неё разностная схема

\begin{equation*}
	L_{\dx}u^{(\dx)} = f^{(\dx)}
\end{equation*}

\noindent Разностная схема () является устойчивой, если существуют такие числа $\Delta > 0$ и $\delta > 0$, что для 
любых $\dx < \Delta$ и $\left|\epsilon^{\dx}\right| < \delta$ разностная задача

\begin{equation*}
	L_{\dx} z^{(\dx)} = f^{(\dx)} + \epsilon^{(\dx)},
\end{equation*}

\noindent полученная добавлением возмущения $\epsilon^{(\dx)}$ к правой части, имеет одно единственное решение $z^{(\dx)}$,
отклоняющееся от решения $u^{(\dx)}$ на величину, удовлетворяющую оценке

\begin{equation*}
	\left|z^{(\dx)} - u^{(\dx)}\right| \le M\left|\epsilon^{(\dx)}\right|,
\end{equation*}

\noindent где $M$ - константа, не зависящая от $\dx$.

Для исследования схем на устойчивость мы будем использовать метод Фурье или, как его ещё называют, 
признак фон Неймана. Рассмотрим простейшую разностную схему и запишем её в каноническом виде:

\begin{equation*}
	\frac{u_{n + 1} - u_n}{\dx} + A u_n = f_n
\end{equation*}

\begin{equation*}
	y_{n + 1} = R_{\dx} y_n + \dx \psi_n,
\end{equation*}

\noindent где $y_n = u_n$, $\psi_n = f_n$  $R_{\dx} = 1 - A\dx$. Важно заметить, что для многоступенчатых схем $\psi_n$
будет отличаться от $f_n$. Для оценки $|R_{\dx}^n|$ воспользуемся собственными значениями оператора $R_{\dx}$, полученными
из уравнения

\begin{equation*}
	det|R_{\dx} - \lambda E| = 0,
\end{equation*}

\noindent тогда получим

\begin{equation*}
	R_{\dx}^n y = \lambda^n y
\end{equation*}
\begin{equation*}
	|R_{\dx}^n y| = |\lambda|^n |y|
\end{equation*}
\begin{equation*}
	|R_{\dx}^n| \ge |\lambda|^n
\end{equation*}

\noindent Получается, что необходимое условие ограниченности $|R_{\dx}^n|$ заключается в том, что все собственные 
значения должны лежать в круге на комплексной плоскости:

\begin{equation*}
	|\lambda| \le 1 + C\dx,
\end{equation*}

\noindent где $C$ - некоторая константа, не зависящая от шага $\dx$.

Устойчивость называется абсолютной, если условие устойчивости выполняется независимо от соотношения
шагов по времени и пространству. Важно заметить, что признак фон Неймана является необходимым условием, но
не достаточным, поэтому мы можем говорить только об условной устойчивости, подразумевающую связь между шагами. 
Это требует от нас исследования схем для каждой конкретной решаемой системы для выяснения этой связи.
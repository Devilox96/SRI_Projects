Прежде чем применять какую-либо схему, необходимо понять условия её применимости в контексте данной задачи,
то есть исследовать схему на сходимость. Теорема Лакса-Филиппова-Рябенького утверждает, что конечно-разностная 
схема, аппроксимирующая решение дифференциального уравнения в частных производных с корректно поставленной 
задачей Коши, сходится к этому решению тогда и только тогда, когда эта схема устойчива.

Введём ошибку схемы как разность между точным и численным решениями в некоторый момент времени $n$ после одного 
расчётного шага $\dx$:

\begin{equation*}
	\delta_{\dx}^n = y_{\dx}^n - y(n \dt)
\end{equation*}

\noindent Тогда условие аппроксимации будет записано в таком виде:

\begin{equation*}
	\lim\limits_{\dx \to 0} \frac{\delta_{\dx}^n}{\dx} = 0
\end{equation*}

\noindent Схема будет иметь порядок аппроксимации $m$, если

\begin{equation*}
	\delta_{\dx}^n = O(\dx^{m + 1}), \dx \to 0
\end{equation*}

Учитывая то, что все схемы, приведённые в работе, удовлетворяют условию аппроксимации, имеет смысл исследовать их
только на устойчивость для определения области их сходимости, следуя теореме Лакса-Филиппова-Рябенького.

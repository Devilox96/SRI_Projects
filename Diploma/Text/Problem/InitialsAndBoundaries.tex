Чтобы задать начальные условия, необходимо определиться, для какой планеты мы решаем задачу. За основу
возьмём Землю, поскольку для неё посчитано и измерено больше параметров. За основу возьмём уравнения
геострофического баланса, поскольку нам важно убрать лишние колебания на ранних расчётных шагах:

\begin{equation}
	\displaystyle
	fv_y = \frac{1}{\rho} \frac{\partial p}{\partial x}
\end{equation}
\begin{equation}
	\displaystyle
	fv_x = -\frac{1}{\rho} \frac{\partial p}{\partial y}
\end{equation}
\begin{equation}
	\displaystyle
	p = \rho g h
\end{equation}

\noindent Так как в нашей модели мы не учитываем сжимаемость газа, то плотность можно вынести из-под знака
дифференциала. Для расчёта высоты возьмём среднюю приблизительную скорость ветра (20$\frac{м}{с}$) и
подставим её в уравнение \ref{eq:6}, чтобы получить градиент вдоль долготы (градиент вдоль широт примем нулём). 
В качестве константы при решении дифференциального уравнения возьмём высоту в 10км, которая будет 
соответствовать значению высоты на широте $45^\circ$. Пользуясь приближением $\beta$-плоскости, разложим параметр 
Кориолиса относительно той же широты. Тогда выражение для высоты примет вид 

\begin{equation}
	\displaystyle
	f(y) = f_0 + \beta (y - y_{45^\circ})
\end{equation}
\begin{equation}
	\begin{split}
		\displaystyle
		h(z) = -\frac{<v>}{g} \int_{y_{45^\circ}}^{z} (f_0 + \beta (y - y_{45^\circ})) dy = 
		\\
		= \frac{<v>}{g} y_{45^\circ} \left(f_0 + \beta\left ( \frac{y_{45^\circ}^2}{2} - y_{45^\circ}^2 \right)\right) -
		\frac{<v>}{g} z \left(f_0 + \beta \left (\frac{z^2}{2} - zy_{45^\circ} \right)\right) =
		\\
		= \frac{<v>}{2g} (f(z) + f_0) (y_{45^\circ} - z) \approx
		\\
		\approx \frac{f(z)<v>}{g}(y_{45^\circ} - z)
	\end{split}	
\end{equation}

\noindent Для распределения скоростей по расчётной области 
подставим полученную высоту в уравнения \ref{eq:5} и \ref{eq:6}. Таким образом мы получаем начальные условия:

\begin{equation}
	\displaystyle
	v_y = \frac{g}{f} \frac{\partial h}{\partial x} = 0
\end{equation}
\begin{equation}
	\displaystyle
	v_x = \frac{g}{f} \frac{\partial h}{\partial y},
\end{equation}

\noindent где $\frac{\partial h}{\partial y}$ можно посчитать простой линейной аппроксимацией

\begin{equation}
	\displaystyle
	\frac{\partial h_{i, j}}{\partial y} = \frac{h_{i, j + 1} - h_{i, j - 1}}{2 \dy}
\end{equation}

\noindent Распределение же компонент магнитного поля получим с помощью калькулятора \cite{MagneticCalc}, 
основанного на WMM модели.
	
При решении данной задачи мы получим не только искомые волны Россби, но и им симметричные, однако направление
движения этих волн строго определяется направлением вращения планеты, поэтому необходимо задать граничные
условия, которые будут способствовать затуханию "лишних" волн. Для этого на левой границе занулим производные 
скоростей вдоль широты (outlet condition), это будет означать, что возмущения, пересекающие данную границу
уходят за пределы расчётной области. В то же время на правой границе зададим периодические условия для скоростей,
поскольку мы ведём расчёт в кольце. Для остальных величин системы периодическими условия будут на обеих
границах. На южной и северной границах зафиксируем начальные условия. Важно заметить, что процесс зануления
производной на границе усложняется с повышением точности схемы, поскольку количество точек, участвующих в
расчётах, увеличивается. Если для расчёта в данном конкретном узле не хватает значений, а дополнительные 
точки (ghost points) зарезервированы под другие узлы, то можно воспользоваться экстраполяцией, получаемой 
разложением функции по Тейлору. Например, экстраполяция по четырём точкам будет иметь вид:

\begin{equation}
	\displaystyle
	f(x_j) = 4f(x_{j + 1}) - 6f(x_{j + 2}) + 4f(x_{j + 3}) - f(x_{j + 4})
\end{equation}
\begin{equation}
	\displaystyle
	f(x_j) = 4f(x_{j - 1}) - 6f(x_{j - 2}) + 4f(x_{j - 3}) - f(x_{j - 4})
\end{equation}
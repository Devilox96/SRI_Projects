TVD (Total Variation Diminishing) схемы \cite{TheoryTVD} - схемы, удовлетворяющие неравенству

\begin{equation*}
	\displaystyle
	TV(u^{n + 1}) \le TV(u^n),
\end{equation*}

\noindent где

\begin{equation*}
	\displaystyle
	TV(u(t)) = \int \left| \frac{\partial u}{\partial x} \right| dx
\end{equation*}

\noindent или в дискретном случае

\begin{equation*}
	\displaystyle
	TV(u^n) = \sum_{j} \left| u^n_{j + 1} - u^n_j \right|
\end{equation*}

Рассмотрим явную схему:

\begin{equation*}
	\displaystyle
	\frac{\partial u}{\partial t} + \frac{\partial X}{\partial x} = 0
\end{equation*}
\begin{equation*}
	\displaystyle
	u^{n + 1}_j = u^n_j - \frac{\dt}{\dx} \left( X^n_{j + \frac{1}{2}} - X^n_{j - \frac{1}{2}} \right)
\end{equation*}

\noindent В общем случае для схем первого порядка поток, например, $X^n_{j + \frac{1}{2}}$, можно записать в виде

\begin{equation*}
	\displaystyle
	X^n_{j + \frac{1}{2}} = \frac{1}{2} \left( X^n_{j + 1} + X^n_j - \psi
	\left( a^n_{j + \frac{1}{2}} \right) \left( u^n_{j + 1} - u^n_j \right) \right),
\end{equation*}

\noindent где $a^n_{j + \frac{1}{2}}$ - аргумент, зависящий от собственных значений оператора $X$, а $\psi$ - 
функция, отвечающая за численную вязкость, вводится в статье \cite{Entropy} Хартеном как

\begin{equation*}
	\displaystyle
	\psi(y) = 
	\begin{cases}
		\displaystyle
		\frac{y^2}{4 \epsilon} + \epsilon, \: |y| < 2 \epsilon
		\\[10px]
		|y|, \qquad |y| \ge 2 \epsilon
	\end{cases},
\end{equation*}

\noindent где величина $\epsilon$ лежит в пределах $\left(0, \frac{1}{2}\right]$

Определив поток, для построения TVD схемы мы можем пойти двумя путями \cite{TVD}.

\begin{itemize}
	\item
	Первый подход заключается в замене потока на его сумму с некоторой функцией $\tilde{X}$, чтобы
	получившийся поток был второго порядка аппроксимации по отношению к $X$. Тогда получим
	
	\begin{equation*}
		\displaystyle
		X^n_{j + \frac{1}{2}} = \frac{1}{2} \left( X^n_{j + 1} + X^n_j + \tilde{X}^n_{j + 1} + \tilde{X}^n_j - 
		\psi \left( a^n_{j + \frac{1}{2}} + \tilde{a}^n_{j + \frac{1}{2}} \right) \left( u^n_{j + 1} - u^n_j \right) \right),
	\end{equation*}
	
	где $a^n_{j + \frac{1}{2}}$ и $\tilde{a}^n_{j + \frac{1}{2}}$ принимают значения
	
	\begin{equation*}
		\displaystyle
		a^n_{j + \frac{1}{2}} =
		\begin{cases}
			\displaystyle
			\frac{X^n_{j + 1} - X^n_j}{u^n_{j + 1} - u^n_j}, \;\; u^n_{j + 1} \ne u^n_j
			\\[10px]
			\displaystyle
			\left( \frac{dX}{du} \right)_j, \qquad u^n_{j + 1} = u^n_j
		\end{cases}
	\end{equation*}
	\begin{equation*}
		\displaystyle
		\tilde{a}^n_{j + \frac{1}{2}} =
		\begin{cases}
			\displaystyle
			\frac{\tilde{X}^n_{j + 1} - \tilde{X}^n_j}{u^n_{j + 1} - u^n_j}, \;\; u^n_{j + 1} \ne u^n_j
			\\[10px]
			\displaystyle
			0, \qquad\qquad\quad\; u^n_{j + 1} = u^n_j
		\end{cases}
	\end{equation*}
	
	\item
	Второй, наиболее распространённый, путь состоит в добавлении к выражению для потока антидиффузионного
	члена, помноженного на некоторую функцию-ограничитель $\phi$, что позволяет контролировать область устойчивости
	полученной схемы. Для простоты рассмотрим линейный поток:
	
	\begin{equation*}
		\displaystyle
		X(u) = au, \: a = const, \: a > 0
	\end{equation*}
	\begin{equation*}
		\displaystyle
		X^n_{j + \frac{1}{2}} = \frac{1}{2} \left( X^n_{j + 1} + X^n_j - a 
		\left( u^n_{j + 1} - u^n_j \right) \right) + \frac{a}{2}
		\left( u^n_{j} - u^n_{j - 1} \right) \phi_j
	\end{equation*}
	
	\noindent Подставим полученный поток в исходную схему:
	
	\begin{equation*}
		\displaystyle
		\theta_j = \frac{u^n_{j} - u^n_{j - 1}}{u^n_{j + 1} - u^n_{j}}, \quad
		\tilde{\theta}_j = \frac{u^n_{j + 1} - u^n_{j}}{u^n_{j} - u^n_{j - 1}}
	\end{equation*}
	\begin{equation*}
		\displaystyle
		u^{n + 1}_j = u^n_j - a \frac{\dt}{\dx} \left( 1 + \frac{1}{2} \phi_j -
		\frac{1}{2} \theta_{j - 1} \phi_{j - 1} \right) \left( u^n_{j} - u^n_{j - 1} \right)
	\end{equation*}
	
	\noindent Существует множество различных ограничителей, среди которых наиболее известные
	
	\noindent Minmod:
	
	\begin{equation*}
		\displaystyle
		\phi_j = minmod \left(1, \tilde{\theta}_j \right)
	\end{equation*}
	
	\noindent Superbee:
	
	\begin{equation*}
		\displaystyle
		\phi_j = max \left( 0, min \left(1, 2 \tilde{\theta}_j \right), min \left(2, \tilde{\theta}_j \right) \right)
	\end{equation*}
	
	\noindent Ван Лир:
	
	\begin{equation*}
		\displaystyle
		\phi_j = \frac{\tilde{\theta}_j + \left| \tilde{\theta}_j \right|}{1 + \tilde{\theta}_j}
	\end{equation*}
\end{itemize}

В данной работе мы не будем заниматься построением TVD схем, а возьмём на вооружение готовые TVD варианты схем
класса Рунге-Кутта, которые обычно используют для интегрирования по времени. Нас интересует второй порядок
аппроксимации. Пусть $L(u)$ - "пространственная часть" \space уравнения, которую мы аппроксимируем тем или
иным способом, тогда "временную часть" \space можно аппроксимировать следующим образом:

\begin{equation*}
	\begin{cases}
		u^{(1)}_j = u^n_j - \dt L \left( u^n_j \right)
		\\[10px]
		u^{n + 1}_j = \frac{1}{2} u^n_j - \dt L \left( u^n_j \right) - \dt L \left( u^{(1)}_j \right)
	\end{cases}
\end{equation*}

\noindent Построение Рунге-Кутта схем разных порядков, а также их TVD-вариантов можно найти в книге
\cite{RungeKuttaSchemes}. Применение TVD схем к уравнениям мелкой воды описано в статье \cite{ShallowWaterTVD}.
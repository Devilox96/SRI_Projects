Как было описано раннее, в разделе 3.1, схемы второго и выше порядков имеют очень серьёзный недостаток -
колебания в областях разрывов производных, что может приводить к неконтролируемому увеличению модуля
собственных значений матрицы усиления и последующей расходимости схемы. Для решения этой проблемы существуют
несколько подходов: использование неосциллирующих схем, про которые будет идти речь в разделе 4, и введение 
искусственной вязкости \cite{ArtifVisc, ShockViscosity, Oscillations}.

Искусственную вязкость можно ввести в виде добавки к потоку \cite{Smoothing}:

\begin{equation*}
	\displaystyle
	\frac{\partial u}{\partial t} + \frac{\partial X}{\partial x} = 0
\end{equation*}
\begin{equation*}
	\displaystyle
	X' = X - \nu \frac{\partial u}{\partial x}
\end{equation*}
\begin{equation*}
	\displaystyle
	\nu = c_\nu \dx^2 \left| \frac{\partial u}{\partial x} \right|,
\end{equation*}

\noindent где $c_\nu$ - коэффициент вязкости. Тогда уравнение с вязкостью приобретает вид

\begin{equation*}
	\displaystyle
	\frac{\partial u}{\partial t} + \frac{\partial X}{\partial x} = \frac{\partial}{\partial x} 
	\left( c_\nu \dx^2 \left| \frac{\partial u}{\partial x} \right| \frac{\partial u}{\partial x} \right)
\end{equation*}

\noindent Аппроксимируем правую часть:

\begin{equation*}
	\begin{split}
	\frac{\partial}{\partial x} 
	\left( c_\nu \dx^2 \left| \frac{\partial u}{\partial x} \right| \frac{\partial u}{\partial x} \right) \approx
	\\
	\approx \frac{\dt}{\dx} \left(
	\left( c_\nu \dx^2 \left| \frac{\partial u}{\partial x} \right| \frac{\partial u}{\partial x} \right)_
	{j + \frac{1}{2}} -
	\left( c_\nu \dx^2 \left| \frac{\partial u}{\partial x} \right| \frac{\partial u}{\partial x} \right)_
	{j - \frac{1}{2}} \right) =
	\\
	= c_\nu \frac{\dt}{\dx} \left(
	\left| u^n_{j + 1} - u^n_j \right| \left( u^n_{j + 1} - u^n_j \right) -
	\left| u^n_j - u^n_{j - 1} \right| \left( u^n_j - u^n_{j - 1} \right)
	\right)
	\end{split}
\end{equation*}

\begin{figure}[htbp]
	\centering
	\includesvg[width=120px]{SVGs/Viscosity}
	\caption{Искусственная вязкость фон Неймана - Рихтмайера}
\end{figure}

\noindent Из конечной формулы и рисунка видно, что выражение в скобках выступает в роли параметра сглаживания:
чем меньше разброс подряд идущих значений, тем меньше сглаживание и, следовательно,  величина искусственной 
вязкости.
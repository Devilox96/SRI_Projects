\begin{figure}[htbp]
	\centering
	\includesvg[width=170px]{SVGs/Richtmyer}
	\caption{Схема Рихтмайера}
\end{figure}

Применимо к нашей системе, в двумерном случае схема Рихтмайера в виде четырёх расчётов промежуточной
ступени и финального расчёта второй ступени. Такие схемы называют двухшаговыми:

\begin{equation}		
	\begin{cases}
		\displaystyle U_{i - 1, j}^{n + \frac{1}{2}} = \frac{1}{4} \left(
		U_{i, j}^{n} + U_{i - 2, j}^{n} + U_{i - 1, j + 1}^{n} + U_{i - 1, j - 1}^{n} \right) -
		\frac{\dt}{2} \left( \frac{X_{i, j}^{n} - X_{i - 2, j}^{n}}{\dx} +
		\frac{Y_{i - 1, j + 1}^{n} - Y_{i - 1, j - 1}^{n}}{\dy} +
		2 S_{i - 1, j}^{n} \right)
		\\[10pt]
		\displaystyle U_{i + 1, j}^{n + \frac{1}{2}} = \frac{1}{4} \left(
		U_{i + 2, j}^{n} + U_{i, j}^{n} + U_{i + 1, j + 1}^{n} + U_{i + 1, j - 1}^{n} \right) -
		\frac{\dt}{2} \left( \frac{X_{i + 2, j}^{n} - X_{i, j}^{n}}{\dx} +
		\frac{Y_{i + 1, j + 1}^{n} - Y_{i + 1, j - 1}^{n}}{\dy} +
		2 S_{i + 1, j}^{n} \right)
		\\[10pt]
		\displaystyle U_{i, j - 1}^{n + \frac{1}{2}} = \frac{1}{4} \left(
		U_{i + 1, j - 1}^{n} + U_{i - 1, j - 1}^{n} + U_{i, j}^{n} + U_{i, j - 2}^{n} \right) -
		\frac{\dt}{2} \left( \frac{X_{i + 1, j - 1}^{n} - X_{i - 1, j - 1}^{n}}{\dx} +
		\frac{Y_{i, j}^{n} - Y_{i, j - 2}^{n}}{\dy} +
		2 S_{i, j - 1}^{n} \right)
		\\[10pt]
		\displaystyle U_{i, j + 1}^{n + \frac{1}{2}} = \frac{1}{4} \left(
		U_{i + 1, j + 1}^{n} + U_{i - 1, j + 1}^{n} + U_{i, j + 2}^{n} + U_{i, j}^{n} \right) -
		\frac{\dt}{2} \left( \frac{X_{i + 1, j + 1}^{n} - X_{i - 1, j + 1}^{n}}{\dx} +
		\frac{Y_{i, j + 2}^{n} - Y_{i, j}^{n}}{\dy} +
		2 S_{i, j + 1}^{n} \right)
	\end{cases}
\end{equation}	
\begin{equation}
	\displaystyle U_{i, j}^{n + 1} = U_{i, j}^{n} - 
	\dt \left( \frac{X_{i + 1, j}^{n + \frac{1}{2}} - X_{i - 1, j}^{n + \frac{1}{2}}}{\dx} +
	\frac{Y_{i, j + 1}^{n + \frac{1}{2}} - Y_{i, j - 1}^{n + \frac{1}{2}}}{\dy} +
	S_{i, j}^{n} \right)
\end{equation}
\begin{equation*}
	\displaystyle
	X_{i, j}^{n} = X(U_{i, j}^{n}), X_{i, j}^{n + \frac{1}{2}} = X(U_{i, j}^{n + \frac{1}{2}})
\end{equation*}
\begin{equation*}
	\displaystyle
	Y_{i, j}^{n} = Y(U_{i, j}^{n}), Y_{i, j}^{n + \frac{1}{2}} = Y(U_{i, j}^{n + \frac{1}{2}})
\end{equation*}
\begin{equation*}
	\displaystyle
	S_{i, j}^{n} = S(U_{i, j}^{n})
\end{equation*}

\noindent На рисунке красным кружком является не только искомым узлом на следующем временном шаге, но и опорным 
для промежуточных расчётом первой ступени, синие кружки выступают опорными узлами для той же ступени. Серые 
кружки - опорные для финального шага. Заметим, что для данной схемы толщина "рамки"\space дополнительных
узлов (ghost points) вокруг сетки расчётной области составляет уже два значения.
Схемы первого порядка имеют самую низкую точность, однако крайне просты в реализации и требуют достаточно
мало ресурсов CPU/GPU. В силу склонности к сглаживанию подобные схемы будут оптимальны в случае, когда 
решение не имеет больших перепадов производных. Распишем некоторые варианты этих схем:

\begin{equation}
	\displaystyle
	\frac{\partial u}{\partial t} + a \frac{\partial u}{\partial x} = 0
\end{equation}

\begin{itemize}
	\item Прямая схема Эйлера
	\begin{equation}
		\displaystyle
		u^{n + 1}_j = u^n_j - \frac{a \dt}{2 \dx} \left( u^n_{j + 1} - u^n_{j - 1} \right)
	\end{equation}
	
	\item Обратная схема Эйлера
		\begin{equation}
		\displaystyle
		u^{n + 1}_j = u^n_j - \frac{a \dt}{2 \dx} \left( u^{n + 1}_{j + 1} - u^{n + 1}_{j - 1} \right)
	\end{equation}
	
	\item Схема Upwind ("Уголок")
	\begin{equation}
		\displaystyle
		u^{n + 1}_j = u^n_j - \frac{a \dt}{\dx} \left( u^n_{j + 1} - u^n_j \right)
	\end{equation}
	
	\item Прямая схема Лакса-Фридрихса
	\begin{equation}
		\displaystyle
		u^{n + 1}_j = \frac{1}{2} \left( u^n_{j + 1} + u^n_{j - 1} \right) - 
		\frac{a \dt}{2 \dx} \left( u^n_{j + 1} - u^n_{j - 1} \right)
	\end{equation}
\end{itemize}

Вышеперечисленные схемы подробно расписаны и сравнены между собой в \cite{SchemesComparison}, мы же в данной
работе будем рассматривать только последнюю из них.
\begin{figure}[htbp]
	\centering
	\includesvg[width=170px]{SVGs/LaxFriedrichs}
	\caption{Схема Лакса-Фридрихса}
\end{figure}

Для нашей системы в двумерном случае данная схема имеет вид
	
\begin{equation}
	\displaystyle U_{i, j}^{n + 1} = \frac{1}{4} \left(
	U_{i - 1, j}^{n} + U_{i + 1, j}^{n} + U_{i, j - 1}^{n} + U_{i, j + 1}^{n} \right) - 		
	\frac{\dt}{2} \left( \frac{X_{i + 1, j}^{n} - X_{i - 1, j}^{n}}{\dx} +
	\frac{Y_{i, j + 1}^{n} - Y_{i, j - 1}^{n}}{\dy} +
	2 S_{i, j}^{n} \right)
\end{equation}

\begin{equation*}
	\displaystyle
	X_{i, j}^{n} = X(U_{i, j}^{n})
\end{equation*}
\begin{equation*}
	\displaystyle
	Y_{i, j}^{n} = Y(U_{i, j}^{n})
\end{equation*}
\begin{equation*}
	\displaystyle
	S_{i, j}^{n} = S(U_{i, j}^{n})
\end{equation*}

\noindent На рисунке красным кружком отмечен искомый узел на следующем временном слое, синие кружки - опорные
узлы схемы. Поскольку для получения искомого узла требуются ближайшие узлы, то для расчёта граничных условий
вокруг сетки расчётной области необходимо создать "рамку"\space одинарной толщины из дополнительных узлов (ghost
points). В случае outlet условия дополнительный узел будет равен предпоследнему на данном направлении.
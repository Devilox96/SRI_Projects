В случае, когда система является нелинейной, метод фон Неймана не может быть напрямую применён,
и требуется линеаризация. Для простоты рассмотрим одномерную систему уравнений мелкой воды:
	
\begin{equation}
	\label{LaxExample}
	\begin{cases}
		\displaystyle 
		\frac{\partial h}{\partial t} + 
		\frac{\partial (hv_x)}{\partial x} = 0 
		\\[10pt]
		\displaystyle 
		\frac{\partial (hv_x)}{\partial t} + 
		\frac{\partial \left( hv_x^2 + \frac{1}{2}gh^2 \right)}{\partial x} = 0
	\end{cases}
\end{equation}
	
\noindent Преобразуем нелинейные слагаемые с помощью введения матрицы Якоби:

\begin{equation*}
	\begin{pmatrix}
		\displaystyle
		\frac{\partial (hv_x)}{\partial x}
		\\[10pt]
		\displaystyle			
		\frac{\partial \left( hv_x^2 + \frac{1}{2}gh^2 \right)}{\partial x}
	\end{pmatrix}
	=
	\begin{pmatrix}
		\displaystyle
		\frac{\partial F_1 (h, hv_x)}{\partial x}
		\\[10pt]
		\displaystyle			
		\frac{\partial F_2(h, hv_x)}{\partial x}
	\end{pmatrix}
	= A \cdot
	\begin{pmatrix}
		\displaystyle
		\frac{\partial h}{\partial x}
		\\[10pt]
		\displaystyle			
		\frac{\partial (hv_x)}{\partial x}
	\end{pmatrix}
\end{equation*}

\noindent Поскольку $h$ и $hv_x$ неортогональны, сделаем замену переменных $q_1 = h, q_2 = hv_x$, тогда

\begin{equation*}
	\begin{pmatrix}
		\displaystyle
		F_1(q_1, q_2)
		\\[10pt]
		\displaystyle			
		F_2(q_1, q_2)
	\end{pmatrix}
	=
	\begin{pmatrix}
		\displaystyle
		q_2
		\\[10pt]
		\displaystyle			
		\frac{q_2^2}{q_1} + \frac{1}{2}gq_1^2
	\end{pmatrix},
	A =
	\begin{pmatrix}
		\displaystyle
		\frac{\partial F_1}{\partial q_1}
		&
		\displaystyle
		\frac{\partial F_1}{\partial q_2}
		\\[10pt]
		\displaystyle			
		\frac{\partial F_2}{\partial q_1}
		&
		\displaystyle
		\frac{\partial F_2}{\partial q_2}
	\end{pmatrix}
	=
	\begin{pmatrix}
		\displaystyle
		0
		&
		\displaystyle
		1
		\\[10pt]
		\displaystyle			
		-\frac{q_2^2}{q_1^2} + gq_1
		&
		\displaystyle
		\frac{2q_2}{q_1}
	\end{pmatrix}
\end{equation*}

\noindent После обратной замены получим систему:

\begin{equation}\label{eq:6}
	\begin{pmatrix}
		\displaystyle
		\partial_t h
		\\[10pt]
		\displaystyle
		\partial_t (hv_x)
	\end{pmatrix}
	+
	\begin{pmatrix}
		\displaystyle
		0
		&
		\displaystyle
		1
		\\[10pt]
		\displaystyle			
		-v_x^2 + gh
		&
		\displaystyle
		2v_x
	\end{pmatrix}
	\begin{pmatrix}
		\displaystyle
		\partial_x h
		\\[10pt]
		\displaystyle
		\partial_x (hv_x)
	\end{pmatrix}
	= 0
\end{equation}

\noindent Для линеаризации необходимо зафиксировать значения матрицы Якоби для данного шага по времени.
Каждое уравнение системы имеет свою ошибку, поэтому, ориентируясь на подход к линейному уравнению, запишем:

\begin{equation*}
	\begin{split}
	\begin{pmatrix}
		\displaystyle
		E^{nt + \dt}
		\\[10pt]
		\displaystyle
		D^{nt + \dt}
	\end{pmatrix}
	= \frac{1}{2}
	\begin{pmatrix}
		\displaystyle
		E^{nt}(e^{ik\dx} + e^{-ik\dx})
		\\[10pt]
		\displaystyle
		V^{nt}(e^{ik\dx} + e^{-ik\dx})
	\end{pmatrix}
	- \frac{\dt}{2\dx}
	\begin{pmatrix}
		\displaystyle
		0
		&
		\displaystyle
		1
		\\[10pt]
		\displaystyle			
		-v_x^2 + gh
		&
		\displaystyle
		2v_x
	\end{pmatrix}
	\begin{pmatrix}
		\displaystyle
		E^{nt}(e^{ik\dx} - e^{-ik\dx})
		\\[10pt]
		\displaystyle
		V^{nt}(e^{ik\dx} - e^{-ik\dx})
	\end{pmatrix}
	=
	\\
	=
	\begin{pmatrix}
		\displaystyle
		E^{nt}cos(k\dx)
		\\[10pt]
		\displaystyle
		V^{nt}cos(k\dx)
		\end{pmatrix}
		- i \frac{\dt}{\dx}
	\begin{pmatrix}
		\displaystyle
		0
		&
		\displaystyle
		1
		\\[10pt]
		\displaystyle			
		-v_x^2 + gh
		&
		\displaystyle
		2v_x
	\end{pmatrix}
	\begin{pmatrix}
		\displaystyle
		E^{nt}sin(k\dx)
		\\[10pt]
		\displaystyle
		V^{nt}sin(k\dx)
	\end{pmatrix}
	=
	\\
	=
	\begin{pmatrix}
		\displaystyle
		E^{nt}cos(k\dx) -i V^{nt} \frac{\dt}{\dx} sin(k\dx)
		\\[10pt]
		\displaystyle			
		-i E^{nt} \frac{\dt}{\dx} (-v_x^2 + gh) sin(k\dx) + V^{nt} (cos(k\dx) - 2iv_x \frac{\dt}{\dx}sin(k\dx)))
	\end{pmatrix}
	\end{split}
\end{equation*}
\begin{equation}
	\begin{pmatrix}
		\displaystyle
		E^{nt + \dt}
		\\[10pt]
		\displaystyle
		D^{nt + \dt}
	\end{pmatrix}
	= G
	\begin{pmatrix}
		\displaystyle
		E^{nt}
		\\[10pt]
		\displaystyle
		D^{nt}
	\end{pmatrix}
\end{equation}
\begin{equation}\label{eq:8}
	G =
	\begin{pmatrix}
		\displaystyle
		cos(k\dx)
		&
		\displaystyle
		-i \frac{\dt}{\dx} sin(k\dx)
		\\[10pt]
		\displaystyle			
		-i \frac{\dt}{\dx} (-v_x^2 + gh) sin(k\dx)
		&
		\displaystyle
		-2iv_x \frac{\dt}{\dx} sin(k\dx) + cos(k\dx)
	\end{pmatrix}
\end{equation}

\noindent Матрица (\ref{eq:8}) называется матрицей усиления (amplification matrix). Модули её собственных
значений должны быть меньше либо равны единице для всех узлов расчётной сетки на данном временном слое, что
обеспечивает устойчивость схемы. В силу того, что мы решаем систему уравнений, собственные значения матрицы $G$
будут зависеть от собственных значений матрицы $A$ из уравнения \ref{eq:6}.

\begin{equation*}
	\begin{split}
		\lambda_j = cos(k\dx) - iv_x \frac{\dt}{\dx} sin(k\dx) \pm i |sin(k\dx)|\sqrt{gh} \frac{\dt}{\dx} = \\
		cos(k\dx) - i \frac{\dt}{\dx} sin(k\dx) (v_x \mp sgn(sin(k\dx)) \sqrt{gh}) = \\
		cos(k\dx) - i \frac{\dt}{\dx} sin(k\dx) (v_x \pm \sqrt{gh})
	\end{split}
\end{equation*}
\begin{equation*}
	|\lambda_j| = \sqrt{cos^2(k\dx) + \frac{\dt^2}{\dx^2} sin^2(k\dx) (v_x \pm \sqrt{gh})^2} \le 1
\end{equation*}
\begin{equation*}
	|\lambda_j| = \sqrt{1 + sin^2(k\dx) \left( \frac{\dt^2}{\dx^2} (v_x \pm \sqrt{gh})^2 - 1 \right)} \le 1
\end{equation*}
\begin{equation*}
	sin^2(k\dx) \left( \frac{\dt^2}{\dx^2} (v_x \pm \sqrt{gh})^2 - 1 \right) \le 0
\end{equation*}
\begin{equation*}
	\frac{\dt}{\dx} |v_x \pm \sqrt{gh}| \le 1
\end{equation*}
\begin{equation}
	\dt \le \frac{\dx}{|v_x \pm \sqrt{gh}|}
\end{equation}

\noindent Обратим внимание на то, что собственные значения матрицы $A$ равны

\begin{equation}
	\lambda_{12} = v_x \pm \sqrt{gh}
\end{equation}